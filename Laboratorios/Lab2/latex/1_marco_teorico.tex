\section{Marco teórico}

\subsection{Procesamiento de señales y sensores de color}

% Explicar brevemente cómo funciona un LDR (Light Dependent Resistor) y su relación entre resistencia e intensidad luminosa.

% Introducir el modelo de color RGB, sus componentes y cómo se representan los colores en un espacio tridimensional.

% Explicar el concepto de calibración y cómo la distancia euclidiana puede emplearse para identificar colores a partir de valores de sensores.

\subsection{Señales PWM y control de servomotores}

% Explicar el principio del PWM (Pulse Width Modulation) y su relación con el control del ángulo en servos.

% Incluir cómo el Timer1 genera una señal PWM con un periodo de 20 ms para mover el servomotor.

\subsection{Reproducción de sonido digital}

% Introducir la relación entre frecuencia y tono musical.

% Explicar cómo los timers generan ondas cuadradas que excitan un buzzer piezoeléctrico.

% Mencionar la noción de melodía digital y uso de estructuras de datos o interrupciones para reproducir secuencias musicales.

\subsection{Interfaz de usuario e interacción hombre–máquina}

% Explicar brevemente qué es una interfaz de usuario (UI) en sistemas embebidos.

% Describir cómo una máquina de estados organiza la interacción entre el usuario y el sistema (teclado, LCD, LEDs, buzzer).

% Introducir la idea de feedback visual y auditivo.

\subsection{Almacenamiento no volátil y EEPROM}

% Describir las características principales de la memoria EEPROM, su capacidad, durabilidad y limitaciones de escritura.

% Explicar su uso para guardar contraseñas o configuraciones persistentes.

\subsection{Planificación temporal y multitarea cooperativa}

% Introducir el concepto de task scheduler o planificador de tareas basado en un contador global (como millis).

% Explicar cómo un solo timer puede manejar varios procesos periódicos sin bloqueos ni polling.