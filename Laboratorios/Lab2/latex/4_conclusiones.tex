\section{Conlusiones}

\subsection{Plotter}
\subsection{Colores}

\begin{itemize}
\item La comunicación por USART fue una limitación para la velocidad del sistema. Podría mejorarse reduciendo la cantidad de datos enviados o usando interrupciones para hacerlo más fluido.
\item Sería útil agregar un modo de calibración donde el usuario pueda registrar los valores de referencia de cada color, adaptando el sistema a distintas condiciones de luz.
\item Se podría usar un servomotor de 360° para aprovechar todos los colores disponibles en la rueda.
\end{itemize}

\subsection{Piano}

\begin{itemize}
\item Sería interesante agregar más canciones al repertorio.
\item Se podría mejorar la coordinación entre las pistas, ya que con el tiempo tienden a desfasarse un poco.
\item También sería bueno incluir más notas en el teclado para ampliar el rango musical.
\end{itemize}

\subsection{Cerradura}

\begin{itemize}
\item Agregar un botón para ver la contraseña escrita antes de confirmar ayudaría al usuario sin afectar la seguridad.
\item Implementar un solenoide real permitiría que el sistema abra y cierre físicamente el cerrojo.
\item En la simulación de PicSimLab el reinicio no siempre funciona correctamente. A veces es necesario usar el botón “DEBUG” o reiniciar el programa. Sin embargo, en el dispositivo real el sistema funciona correctamente.
\end{itemize}