\section{Introducción}

% Este laboratorio se enmarca dentro del curso de Microcontroladores, y tiene como objetivo aplicar los conocimientos adquiridos sobre arquitectura AVR y programación en C para el desarrollo de sistemas embebidos funcionales. A diferencia del laboratorio anterior, que se centró en el manejo básico de periféricos y estructuras de control, en esta segunda instancia se busca integrar varios módulos interactivos en un entorno de tiempo real.

\subsection{Propósito del laboratorio}

% El objetivo principal es diseñar, programar e integrar distintos subsistemas sobre el microcontrolador ATmega328P, demostrando el uso combinado de entradas analógicas y digitales, comunicación serial, control por PWM, almacenamiento no volátil y planificación de tareas. Cada módulo implementado aborda un conjunto diferente de competencias, permitiendo comprender el funcionamiento coordinado del hardware y el software en un entorno embebido.

\subsection{Descripción general de los módulos}

% El laboratorio se divide en cuatro desarrollos principales:

    % Plotter: sistema de control cartesiano para movimiento de ejes y trazado de figuras.

    % Colores: sensor RGB con medición por LDR, control de servomotor y replicación visual en una tira LED.

    % Piano: reproducción de melodías mediante timers, interrupciones y manejo de frecuencias.

    % Cerradura electrónica: sistema de acceso con contraseña almacenada en EEPROM, teclado matricial y pantalla LCD.

\subsection{Competencias desarrolladas}

% A través de estos proyectos se ejercitan competencias fundamentales en ingeniería mecatrónica, tales como:

    % Programación estructurada en lenguaje C sobre microcontroladores AVR.

    % Uso de interrupciones y temporizadores para tareas concurrentes.

    % Aplicación de PWM para control de actuadores.

    % Implementación de comunicación serial (USART) y almacenamiento persistente (EEPROM).

    % Diseño de máquinas de estado para gestionar la interacción con el usuario.

\subsection{Cierre o enfoque global}

% En conjunto, el laboratorio permite afianzar la comprensión del funcionamiento interno del ATmega328P y su ecosistema de periféricos, aplicando conceptos de electrónica digital, control en tiempo real y diseño de interfaces. El resultado es un conjunto de sistemas autónomos capaces de procesar información, interactuar con el entorno y ofrecer respuestas predecibles y seguras.