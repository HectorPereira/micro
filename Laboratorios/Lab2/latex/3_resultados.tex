\section{Resultados}
% En esta sección se presentan los resultados obtenidos durante la implementación y prueba de los distintos módulos desarrollados: el sistema de reconocimiento de colores, el piano electrónico, la cerradura con contraseña y el control del plotter. Cada módulo fue verificado individualmente en simulación y en montaje físico, registrando su funcionamiento y las principales observaciones.

\subsection{Resultados por módulo}

\subsubsection{Plotter}
% Explicá cómo se verificó el movimiento (por ejemplo, ejecución de trayectorias simples o figuras predefinidas).

% Mostrá fotos o diagramas de movimiento, si tenés.

% Comentá observaciones: velocidad, precisión, limitaciones por USART.

% 💬 Ejemplo:

% El plotter ejecutó correctamente trayectorias lineales y circulares generadas desde un script en Python. Se comprobó que, al reducir la tasa de envío por USART, la respuesta del sistema fue más fluida y sin interrupciones perceptibles.

\subsubsection{Colores}
% Mostrá la tabla o los valores obtenidos por el LDR para cada color calibrado.

% Incluí una figura de la tira LED mostrando el color detectado o el ángulo del servo.

% Mencioná el comportamiento: estabilidad, sensibilidad a la luz ambiente, etc.

% 💬 Ejemplo:

% El sistema logró identificar correctamente los colores básicos (rojo, verde, azul, amarillo, violeta y blanco). Los valores de tensión medidos variaron entre 0,4 V y 3,2 V según la intensidad reflejada. Se observó que la calibración inicial mejora la repetibilidad, reduciendo errores al repetir la medición sobre el mismo color.

% ----------------------------------------- RAMBLINGS

% Reconocer colores
% Como se reconoce un color
% Un color se puede descomoner en otros colores
% Colores primarios
% Colores CYMK
% Colores RGB
% Un LDR no puede medir componentes
% Un LDR solo mide luz reflejada
% Si iluminamos con luz blanca el LDR mide la cantidad de luz blanca que rebota
% Seria como tener vision en blanco y negro
% Algunos colores serian dificiles de identificar

% Si iluminamos con otro color el LDR ve la cantidad de luz que refleja de ese color
% Si iluminamos con rojo, luego verde, y luego azul el LDR puede identificar 
% colores con 3 grados de libertad como el ojo humano.

% Se utiliza un led RGB para iluminar con estos colores

% El LDR no responde la misma manera a todos los tipos de luz. 
% Los valores medidos no se van a correlacionar de manera directa
% Con las componentes tradicionales de cada color.
% Ademas de que existen variaciones de emision de luz entre los mismos colores del led

% Sin embargo es suficiente si lo calibramos antes

% Utilizando un divisor de voltaje se conecta el ldr a un pin analogico del arduino

% Se ilumina con rojo verde y azul para cada color y se recopilan los valores medidos por el ADC

% Se mapean los 6 colores de la hoja para tomar como referencias.

% Para determinar el color el cual el LDR esta apuntando
% se calcula cual es el valor más próximo tomando
% en cuenta los 3 componentes como ejes de libertad
% realizando un calculo de distancia cartesiana.
% A2 + B2 + C2 = D2

% En este caso rojo verde y azul representan un eje en un espacio tridimensional. 
% Los colores calibrados representan vectores. Y el vector de medición se 
% mueve a lo largo de todo este espacio adquiriendo diferentes valores rgb.

% De manera ciclica se mide la distancia del vector medicion con respecto 
% al resto de vectores mapeados para identificar cual es el vector que 
% mas se aproxima al vector medición

% Problemas
% Como medir colores con presicion
% Como reconocer colores precalibrados
% Como controlar un servomotor
% Como controlar una tira LED

% Funcionaldiades
% Salida de info por USART
% Color indicado por angulo en servomotor
% Color indica

El proceso de reconocimiento de colores se fundamenta en la descomposición de cada color en sus componentes primarias dentro del modelo RGB (\textit{Red}, \textit{Green}, \textit{Blue}). Un color puede representarse como la combinación ponderada de estas tres intensidades, lo que permite su descripción en un espacio tridimensional.  

\vspace{1em}

Sin embargo, el sensor fotoresistivo (LDR, \textit{Light Dependent Resistor}) utilizado en el sistema no distingue de forma individual las componentes cromáticas del espectro visible; únicamente mide la intensidad total de luz reflejada sobre su superficie. Si la iluminación se realiza con luz blanca, el sensor solo entrega una lectura proporcional a la cantidad total de luz reflejada, sin discriminar su composición espectral. Este fenómeno equivale a una percepción en escala de grises, dificultando la identificación precisa de colores.

\vspace{1em}

Para resolver esta limitación, se empleó un diodo emisor de luz RGB como fuente de iluminación controlada. Al iluminar secuencialmente la superficie con luz roja, verde y azul, el sistema obtiene tres mediciones independientes mediante el LDR. Dichos valores, convertidos por el módulo ADC (\textit{Analog-to-Digital Converter}) del microcontrolador ATmega328P, representan las coordenadas $(R, G, B)$ de un punto dentro de un espacio de color tridimensional.

\vspace{1em}

Dado que tanto la respuesta espectral del LDR como la emisión de los LED presentan variaciones no lineales, los valores obtenidos no son directamente proporcionales a las componentes RGB teóricas. No obstante, el sistema logra resultados estables mediante un proceso de calibración inicial, donde se iluminan sucesivamente los colores de referencia (rojo, verde, azul claro, violeta, morado, amarillo y blanco) y se almacenan sus valores de conversión analógica-digital.

\vspace{1em}

Cada color de referencia calibrado se modela como un vector fijo en dicho espacio. Para identificar un color, se calcula la distancia euclidiana entre el vector de medición y cada uno de los vectores de referencia almacenados:

\begin{equation}\label{eq:distancia_color}
D = \sqrt{(R_m - R_i)^2 + (G_m - G_i)^2 + (B_m - B_i)^2}
\end{equation}

donde $(R_m, G_m, B_m)$ representan los valores medidos por el sensor y $(R_i, G_i, B_i)$ corresponden a los valores calibrados de cada color de la hoja de referencia. El color identificado será aquel cuya distancia $D$ sea mínima.



\vspace{1em}

El circuito se implementó mediante un divisor resistivo conectado a una de las entradas analógicas del microcontrolador. El color identificado se replica visualmente en la tira de LED WS2812, y el servomotor se posiciona en el ángulo correspondiente al color detectado dentro de la hoja de referencia, integrando así un sistema de selección e identificación de color completamente automatizado.












\subsubsection{Piano}

% Mostrá capturas de los menús USART, las canciones reproducidas o las frecuencias generadas.

% Explicá el uso de interrupciones para reproducir melodías sin detener el programa.

% Comentá si el sonido fue limpio, si hubo desfase, o cómo responden los botones.

% 💬 Ejemplo:

% El piano electrónico logró reproducir correctamente las melodías “Cha-La Head-Cha-La” y “Still Alive”. Cada nota se generó a partir del temporizador 0 configurado en modo CTC. La reproducción simultánea de pistas A y B mostró un leve desfase tras varios segundos de ejecución, atribuible a pequeñas diferencias en el conteo de interrupciones.

% ---------------------------- RAMBLINGS

% Sonido
% Que es el sonido
% Ondas
% Difrentes tipos de ondas
% Como hacemos sonido
% Buzzer
% Onda cuadrada
% Como funciona el buzzer
% Piezo
% Como lo hacemos sonar con el atmega
% Variacion de frecuencia
% Como hacemos otro sonido
% Cambiamos la frencuencia
% Como hacemos musica
% Teoria musical basica
% Mapeo de notas
% Una nota es una frecuencia
% Una frecuencia durante un tiempo es una nota
% Como guardamos musica
% Nota musical: frencuencia, tiempo encendido, tiempo apagado
% Secuencia de notas musicales --- musica
% La musica se compone de varios instrumentos en paralelo
% Como reproducimos varios instrumentos
% Evitar polling --- Utilizar timers

% Problemas
% Como reproducir sonido
% Como representar una nota musical
% Como reproducir pistas en paralelo

% Funcionalidades
% Control por USART
% Piano fisico


El sonido es una onda mecánica que se propaga a través de un medio elástico, producto de variaciones periódicas de presión. Estas ondas pueden clasificarse según su forma en senoidales, cuadradas, triangulares o diente de sierra, dependiendo del patrón temporal de oscilación. En los sistemas digitales, las señales más sencillas de generar son las ondas cuadradas, donde el voltaje alterna entre dos niveles definidos, representando los estados lógicos alto y bajo del microcontrolador.

\vspace{1em}

Para producir sonido de manera electrónica, se emplean dispositivos piezoeléctricos denominados \textit{buzzers}. Estos transductores convierten la energía eléctrica en vibraciones mecánicas audibles. Cuando el microcontrolador aplica una señal cuadrada a la entrada del buzzer, el material piezoeléctrico se deforma y contrae periódicamente, generando un sonido cuya frecuencia está directamente relacionada con la frecuencia de la señal de entrada.

\vspace{1em}

En el microcontrolador ATmega328P, esta señal se genera mediante el uso de los temporizadores internos (\textit{timers}), configurados para producir una onda cuadrada en un pin de salida. Al modificar la frecuencia de conmutación, es posible variar el tono percibido, ya que la frecuencia del sonido determina su altura musical. De esta manera, cada nota puede asociarse a una frecuencia específica según la escala temperada. Por ejemplo, la nota \textit{La4} corresponde a una frecuencia de 440\,Hz, mientras que \textit{Do4} equivale aproximadamente a 262\,Hz.

\vspace{1em}

Una nota musical puede representarse computacionalmente mediante dos parámetros fundamentales: la frecuencia de oscilación, la duración temporal durante la cual se mantiene activa y la duración inactiva o en silencio. Cuando el buzzer emite una secuencia ordenada de notas, cada una con su tiempo de activación y silencio, se obtiene una melodía. En términos programáticos, una melodía se define como un conjunto de estructuras de datos que contienen pares \texttt{(frecuencia, tiempo\_on, tiempo\_off)}, los cuales son procesados secuencialmente para generar la música deseada.

\vspace{1em}

La reproducción simultánea de varios instrumentos en un entorno digital requiere el manejo de múltiples secuencias de notas en paralelo. Para lograrlo sin recurrir al uso de ciclos de espera activos (\textit{polling}), se utilizan interrupciones asociadas a los temporizadores. De este modo, el microcontrolador puede controlar la duración y el inicio de cada nota de forma independiente y precisa, optimizando el uso del procesador y permitiendo la ejecución de tareas concurrentes, como la lectura de entradas o la comunicación serial, mientras la música continúa sonando de manera autónoma.

\vspace{1em}

En síntesis, el sistema desarrollado utiliza un buzzer piezoeléctrico controlado por los temporizadores del ATmega328P para reproducir notas musicales definidas por su frecuencia y duración. A través de la programación de secuencias almacenadas en memoria, es posible interpretar melodías completas y gestionar la reproducción de distintas canciones sin intervención del usuario durante la ejecución.






\subsubsection{Cerradura}

% Mostrá el flujo de la interfaz (mensajes en LCD, ingreso de contraseña, cambio de clave, activación de alarma).

% Si podés, mostrás una tabla con los estados de la máquina y qué hace cada uno.

% Comentá observaciones de funcionamiento en simulación vs físico.

% 💬 Ejemplo:

% El sistema respondió correctamente al ingreso de contraseñas válidas e inválidas. Tras tres intentos fallidos, se activó la alarma visual y sonora. El cambio de contraseña fue exitoso y persistió tras apagar el microcontrolador, confirmando la correcta escritura en EEPROM. En PicSimLab, el teclado dejó de responder tras reiniciar, pero en el hardware físico el funcionamiento fue estable.

% -------------------------- RAMBLINGS

% Cerradura
% Que tiene que hacer
% Candado cerrado
% Contraseña correcta
% Candado abierto
% Sino 3 veces alarma

% Cambiar contraseña
% Contra actual
% Contra nueva
% Guaradado

% Contraseñas tienen que ser entre 4 y 6 digitos

% Problema
% Una interfaz de usuario
% Que es una interfaz
% Entradas y salidas
% El usuario hace algo y la interfaz muestra algo
% Es la parte del programa o sistema que se encarga
% de hablar con el usuario.
% Diferentes acciones muestran o hacen diferentes cosas
% Maquina de estados --- Transiciones de estado.
% Problema: Reproducir sonidos, hacer debouncing, manejar alarmas, pantalla LCD. 
% No existen suficientes timers
% Solucion: tareas o tasks
% Utilizar un contador de tiempo interno de 32 bits
% junto con un solo timer para manejar tareas en paralelo

% La tarea se ejecuta rapidamente en el main y compara el timepo que paso con 
% el tiempo especificado para una acción especifica.

% Un solo timer. Muchos delays sin polling.

% Otro problema
% Contraseña debe ser guardada aunque el sistema esté apagado
% Solucion EEPROM
% Funcionamiento de la eeprom
% limitaciones de escritura de EEPROM
% Como leer
% Como escribir

% Otro problema
% Teclado matricial
% Usa menos pines, uno por fila uno por columna
% Como identificar botones presionados
% Respuesta multiplexado
% Filas funcionan como salidas
% Columnas funcionan como entradas (pullup interno)
% Al solamente apagar una fila a la vez (0) 
% va a hacer que el pin de esa columna se vuelva 0
% Funcion retorna el caracter que representa esa fila y columna
% Listo.


El sistema de cerradura electrónica desarrollado tiene como objetivo controlar el acceso mediante una contraseña numérica almacenada en memoria no volátil. Inicialmente, el dispositivo se encuentra en estado de \textit{candado cerrado}. Cuando el usuario ingresa la contraseña correcta, el sistema cambia al estado de \textit{candado abierto}, permitiendo el acceso. En caso de introducir una contraseña incorrecta tres veces consecutivas, se activa una alarma acústica a través de un buzzer, indicando un intento de acceso no autorizado.

El sistema también permite modificar la contraseña almacenada. Para ello, el usuario debe ingresar primero la contraseña actual y, tras su validación, definir una nueva contraseña de entre cuatro y seis dígitos. Esta nueva clave se almacena permanentemente en la memoria EEPROM del microcontrolador, garantizando su persistencia incluso después de un apagado o reinicio del sistema.

\vspace{1em}

\paragraph*{Interfaz de usuario e interacción}

La cerradura dispone de una interfaz de usuario compuesta por una pantalla LCD de 16x2, un teclado matricial 4x4 y tres indicadores luminosos (LED verde, LED rojo y alarma sonora). En este contexto, la interfaz constituye el medio de comunicación entre el usuario y el sistema, mostrando mensajes informativos y recibiendo acciones por medio del teclado. Cada interacción del usuario genera una respuesta visual o acústica distinta, representando así un flujo de diálogo entre ambos.

El sistema se diseñó siguiendo una lógica de \textit{máquina de estados finitos}, en la que cada modo de operación (menú principal, ingreso de contraseña, cambio de clave, acceso autorizado, alarma, etc.) representa un estado. Las transiciones entre estados se producen en función de las acciones del usuario y las condiciones del sistema. Este enfoque facilita la gestión de comportamientos complejos, simplifica el control del flujo de ejecución y mejora la legibilidad del código.

\vspace{1em}

\paragraph*{Teclado matricial y detección de teclas}

El ingreso de datos se realiza mediante un teclado matricial 4x4 que combina filas y columnas, optimizando el uso de pines del microcontrolador. Las filas se configuran como salidas y las columnas como entradas con resistencias de \textit{pull-up} activadas. El proceso de lectura consiste en activar una fila a nivel bajo (0) mientras las demás permanecen en alto (1); si alguna tecla de esa fila se encuentra presionada, la columna correspondiente cambia a nivel bajo, permitiendo identificar la intersección entre fila y columna.  

Cada tecla se asocia a un carácter según su posición dentro de la matriz, lo que permite retornar el valor numérico o simbólico correspondiente. Para evitar falsas detecciones debido al rebote mecánico de los contactos, se aplica una rutina de \textit{debouncing} por software basada en pequeños retardos temporizados.

\vspace{1em}

\paragraph*{Gestión temporal y concurrencia de tareas}

Durante el funcionamiento, el sistema debe ejecutar múltiples tareas de forma paralela: reproducción de sonidos, manejo de alarmas, parpadeo de LEDs, lectura del teclado y actualización de la pantalla LCD. Sin embargo, el ATmega328P dispone de un número limitado de temporizadores, por lo que no es posible asignar un temporizador independiente a cada tarea.

Para resolver esta limitación, se implementó un esquema de ejecución basado en un \textit{planificador de tareas} o \textit{task scheduler} de propósito general. Se utiliza un solo temporizador configurado para generar interrupciones periódicas, incrementando un contador global de 32 bits que actúa como referencia temporal (en milisegundos). Cada tarea compara el tiempo actual con el instante programado de su próxima ejecución, y si se cumple el intervalo, se ejecuta la acción correspondiente. De este modo, múltiples eventos temporizados pueden coexistir sin emplear retardos bloqueantes ni técnicas de \textit{polling}.

\vspace{1em}

\paragraph*{Almacenamiento persistente de la contraseña}

Para asegurar que la contraseña permanezca almacenada tras un apagado, se utiliza la memoria EEPROM interna del ATmega328P. Esta memoria no volátil permite conservar los datos durante décadas sin alimentación eléctrica, aunque posee un número limitado de ciclos de escritura (aproximadamente $10^5$ operaciones por celda). Por ello, el sistema únicamente escribe en EEPROM cuando el usuario confirma un cambio de contraseña, minimizando el desgaste de la memoria.

La lectura y escritura se realizan mediante las funciones de la biblioteca estándar \texttt{avr/eeprom.h}, que permiten transferir cadenas de caracteres directamente desde y hacia direcciones específicas de la EEPROM. Así, el sistema recupera la contraseña almacenada al inicio del programa y la compara con la ingresada por el usuario durante la operación normal.

\vspace{1em}

\paragraph*{Resumen de funcionamiento}

En conjunto, la cerradura electrónica combina la interacción mediante teclado y pantalla LCD, la gestión de tareas en tiempo real y el almacenamiento persistente de datos, logrando un sistema confiable y autónomo. El uso de una máquina de estados y un planificador temporal basado en un único temporizador permite controlar de manera eficiente múltiples procesos simultáneos sin bloquear la ejecución principal del programa.








\subsection{Evidencias gráficas y mediciones}
% Capturas del LCD, fotos del montaje, gráficos de lectura ADC, diagramas de flujo, etc.

% Si mediste tiempos, tensiones o corrientes, ponelos en tabla o gráfico simple.

\subsection{Comentario general}
% Un breve cierre que enlace con la sección de conclusiones:

	% En general, todos los módulos cumplieron los objetivos planteados. Se lograron lecturas estables, respuestas correctas y sincronización entre las distintas tareas, demostrando la integración de hardware y software en un entorno embebido con recursos limitados.
